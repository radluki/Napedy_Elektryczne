\documentclass[12pt]{article}


\usepackage[utf8]{inputenc}
\usepackage[T1]{fontenc}
\usepackage[polish]{babel}
\usepackage{graphicx}

\usepackage{float}

\title{Sprawozdanie z ćwiczenia A1}
\author{ 
Dawid Legutki \and Piotr Merynda \and Damian Paciuch \and Maciej Podsiadło \and Łukasz Radzio}
\date{Data ćwiczenia: 13.04.2015}

\newtheorem{Def}{Definicja}

\newcommand{\LT}[1]{\mathcal{L}\{#1\}}
\newcommand{\iLT}[1]{\mathcal{L}^{-1}\{#1\}}
\newcommand{\ZMa}{12}
\newcommand{\obrazek}[2]
{
	\begin{figure}[H]
	\centering
	\includegraphics[width=#1 cm]{#2}
	\end{figure}
}
\newcommand{\obrazeko}[3]
{
	\begin{figure}[H]
	\centering
	\includegraphics[width=#1 cm]{#2}
	\caption{#3}
	\end{figure}
}

\begin{document}
\maketitle

\section{Charakterystyka mechaniczna}

\obrazek{12}{Wykresy/Mechaniczna}
\input{Wykresy/Mechaniczna.txt}

\section{Charakterystyka z częścią prądnicową}
\obrazek{12}{Wykresy/Kloss}
\input{Wykresy/Kloss.txt}


\section{Rozruch trójkąt-gwiazda}
\obrazek{12}{Wykresy/trojkat-gwiazda}
\input{Wykresy/trojkat-gwiazda.txt}









\end{document}
