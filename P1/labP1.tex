\documentclass[12pt]{article}


\usepackage[utf8]{inputenc}
\usepackage[T1]{fontenc}
\usepackage[polish]{babel}
\usepackage{graphicx}

\usepackage{float}

\title{Sprawozdanie z ćwiczenia P1}
\author{ 
Dawid Legutki \and Piotr Merynda \and Damian Paciuch \and Maciej Podsiadło \and Łukasz Radzio}
\date{Data ćwiczenia: 30.03.2015}

\newcommand{\obrazek}[2]
{
	\begin{figure}[H]
	\centering
	\includegraphics[width=#1 cm]{#2}
	\end{figure}
}
\newcommand{\obrazeko}[3]
{
	\begin{figure}[H]
	\centering
	\includegraphics[width=#1 cm]{#2}
	\caption{#3}
	\end{figure}
}

\begin{document}
\maketitle

\section{Dane znamionowe silników}
	\obrazek{6}{tabele/DZ_obcowzbudny}
	\obrazek{6}{tabele/DZ_szeregowy}
\section{Charakterystyki silnika obcowzbudnego}

	\obrazek{12}{wykresy/Obcowzbudny}
	\input{Opis/wnioski_obcowzbudny.txt}
	
\section{Charakterystyki silnika szeregowego}

	\obrazek{12}{wykresy/Szeregowy}
	\input{Opis/wnioski_szeregowy.txt}
	
	
	\subsection{Wyprowadzenie wzoru nr 1}
	Równania silnika szeregowego, przy założeniu że strumień jest wprost proporcjonalny do prądu:
	\begin{equation}
	U=E+(R_d+R_r)I
	\end{equation}
	\begin{equation}
	E=c\Phi n=c_2 In
	\end{equation}
	Po podstawieniu wzoru nr 3 do 2 otrzymuje się wzór na charakterystykę:
	\begin{equation}
	n=\frac{U-(R_d+R_r)I}{c_2 I}=\frac{U}{c_2 I} - \frac{R_d+R_r}{c_2}
	\end{equation}
	\subsection{Weryfikacja modelu}
	\obrazek{12}{wykresy/U150Rd0}
	\obrazek{12}{wykresy/U120Rd0}
	\paragraph{Obserwacje}
		\begin{itemize}
			\item prędkość obrotowa jest wprost proporcjonalna do odwrotności prądu
			\item Zastanawiający jest fakt, że współczynniki b prostych aproksymujących są większe od zera. Jest to niezgodne z wzorem nr 1.
		\end{itemize}
	\obrazek{12}{wykresy/U150Rd1}
	\paragraph{Obserwacje}
		\begin{itemize}
			\item Dołączenie szeregowe dodatkowej rezystancji powoduje obniżenie charakterystyki. 
			\item Współczynnik \textbf{b} jest ujemny i mniejszy niż dla $R_d=0$
			\item Niezgodna ze wzorem jest zmiana współczynnika \textbf{a} prostej aproksymującej. Dla $R_d=0$ wynosił on $16632$, a dla $R_d>0$ $17356$. Różnica jest na poziomie 5\%.
		\end{itemize}
	\paragraph{Wnioski}
		\begin{itemize}
			\item Wzór nr 1 nie opisuje silnika szeregowego w sposób dokładny.
			\item Jest to spowodowane przyjętym założeniem: $\Phi = c_3 I$, gdzie $c_3 = const$
			\item W rzeczywistości strumień się nasyca i jego zależność od prądu jest coraz słabsza ze wzrostem prądu.
		\end{itemize}
\section{Charakterystyka rozruchowa silnika szeregowego}

	\obrazek{12}{wykresy/Rozruchowa}
	\input{Opis/wnioski_rozruchowa.txt}



\end{document}