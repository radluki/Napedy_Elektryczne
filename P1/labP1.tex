\documentclass[12pt]{article}


\usepackage[utf8]{inputenc}
\usepackage[T1]{fontenc}
\usepackage[polish]{babel}
\usepackage{graphicx}

\usepackage{float}

\title{Sprawozdanie z ćwiczenia P1}
\author{ 
Dawid Legutki \and Piotr Merynda \and Damian Paciuch \and Maciej Podsiadło \and Łukasz Radzio}
\date{Data ćwiczenia: 30.03.2015}

\newcommand{\obrazek}[2]
{
	\begin{figure}[H]
	\centering
	\includegraphics[width=#1 cm]{#2}
	\end{figure}
}
\newcommand{\obrazeko}[3]
{
	\begin{figure}[H]
	\centering
	\includegraphics[width=#1 cm]{#2}
	\caption{#3}
	\end{figure}
}

\begin{document}
\maketitle

\section{Dane znamionowe silników}
	\obrazek{6}{tabele/DZ_obcowzbudny}
	\obrazek{6}{tabele/DZ_szeregowy}
\section{Charakterystyki silnika obcowzbudnego}

	\obrazek{12}{wykresy/Obcowzbudny}
	\input{Opis/wnioski_obcowzbudny.txt}
	
\section{Charakterystyki silnika szeregowego}

	\obrazek{12}{wykresy/Szeregowy}
	\input{Opis/wnioski_szeregowy.txt}
	
\section{Charakterystyka rozruchowa silnika szeregowego}

	\obrazek{12}{wykresy/Rozruchowa}
	\input{Opis/wnioski_rozruchowa.txt}



\end{document}